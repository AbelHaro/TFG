\documentclass[12pt,a4paper]{report}

% Packages
\usepackage[utf8]{inputenc}
\usepackage[spanish]{babel}
\usepackage{graphicx}
\usepackage{amsmath}
\usepackage{cite}
\usepackage{hyperref}

% Document information
\title{Título del Trabajo de Fin de Grado}
\author{Tu Nombre}
\date{\today}

\begin{document}

% Title page
\maketitle

% Abstract
\begin{abstract}
Este es un resumen del trabajo. Aquí se describe brevemente el contenido, 
la metodología utilizada y las conclusiones principales.
\end{abstract}

% Table of contents
\tableofcontents
\newpage

% Introduction
\chapter{Introducción}
\section{Motivación}
Este es un ejemplo de texto en la sección de motivación.

\section{Objetivos}
Los objetivos de este trabajo son:
\begin{itemize}
    \item Primer objetivo
    \item Segundo objetivo
    \item Tercer objetivo
\end{itemize}

% Chapter example
\chapter{Marco Teórico}
\section{Sección de ejemplo}
Aquí va el contenido de la sección.

% Figure example
\begin{figure}[h]
    \centering
    \caption{Descripción de la figura}
    \label{fig:ejemplo}
\end{figure}

% Table example
\begin{table}[h]
    \centering
    \caption{Tabla de ejemplo}
    \begin{tabular}{|c|c|c|}
        \hline
        Columna 1 & Columna 2 & Columna 3 \\
        \hline
        Dato 1 & Dato 2 & Dato 3 \\
        Dato 4 & Dato 5 & Dato 6 \\
        \hline
    \end{tabular}
    \label{tab:ejemplo}
\end{table}

% Bibliography
\chapter{Bibliografía}
\begin{thebibliography}{99}
    \bibitem{referencia1} Autor, A. (Año). Título del libro. Editorial.
    \bibitem{referencia2} Autor, B. (Año). Título del artículo. Revista, Volumen(Número), páginas.
\end{thebibliography}

\end{document}